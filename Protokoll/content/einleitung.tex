\section{Einleitung}
\label{sec:Einleitung}
In diesem Versuch werden kristalline Festkörper mit der Debye-Scherrer Methode untersucht. 
Der größte Teil der festen Materie ist kristallin. 
Sie zeichnet sich durch eine räumlich periodische Gitterstruktur aus, die sich makroskopisch fortsetzt. 
Die Unterschiede in der Struktur äußern sich in Anisitropen Eigenschaften der Gitterstruktur  und des Atomaren Aufbaus, wie zum Beispiel Elastizität und Permeabilität, die durch Tensor und Vektorfelder im Kristall hervorgerufen werden.\\
Es ist zwischen Einkristallen und Polykristallinen zu unterscheiden. 
Metalle sind in der Regel polykristallin, so wie fast alle natürlich vorkommenden Kristalline.
Die daraus resultierenden makroskopischen Eigenschaften sind isotrop und ergeben sich aus der Mittelung aller Kristallrichtungen im Festkörper.
Zur genauen Untersuchung der periodischen Struktur und inneren Ordnung sind Einkristalle oder einzelne Kristallite notwendig.  
Diese werden mit Röntgen-Strahlung, welche eine Wellenlänge im Bereich Angström besitzt, bestrahlt, damit diese an dem Festkörper gebeugt werden kann.
Ein Angström entspricht der Größenordnung der Gitterabstände der Atome, welche mit der Debye-Scherrer Methode bestimmt werden soll.	